% This file was created automatically from reps.msk.
% DO NOT EDIT!
\Chapter{Ordered Signatures}

This chapter contains some methods for ordered signatures in ordinary
difference sets. Unfortunately, these methods are not as comfortable
as those for unordered signatures. The reason for this is simply that
I didn't have any time to tie them together to high-level functions.
If you need help here, don't hesitate to contact me.


%%%%%%%%%%%%%%%%%%%%%%%%%
\Section{Definition}

Let $R \subseteq G$ be a (partial) ordinary difference set (for
definition see "Introduction"). Let $U\leq G$ be a normal subgroup and
$C=\{g_1,\dots, g_{|G:U|}\}$ be a system of representatives of $G/U$.

As in "The Coset Signature" we may define the coset signature of $R$
relative to $U$.

Let $U=g_1,\dots,g_{|G:U|}$ be an enumeration of $G/U$. An
``admissible ordered signature'' for $U$ is a tuple
$(v_1,\dots,v_{|G:U|})$ such that 

$$
\matrix{
\sum v_i=k\cr
\sum v_i^2=\lambda(|U|-1)+k\cr
\sum_j v_j v_{ij}=
	 \lambda(|U|-1)&{\rm for }\   g_i\not\in U}
$$ 

holds where we index the $v_i$ by elements of $G/U$, so $v_i=v_{g_i}$
and write $v_{ij}=v_{g_ig_j}$. Observe that the third equation is a
restriction on the ordering of the tuple $(v_1,\dots,v_{|G:U|})$. If
$v$ is an admissible ordered signature, then the multiset of $v$ is an
unordered signature.

Getting ordered admissible signatures from unordered ones can be done
by taking all permutations of the unordered signature and verifying
the above equations. Obviously, this method isn't very satisfying
(nevertheless, the methods for testing unordered signatures from
section "The Coset Signature" do this to find out if there is an
ordered signature at all. Except that they stop when they find an
ordered signature).

For ordinary difference sets in extensions of semidirect products of
cyclic groups, ordered signatures may be calculated a lot easier (see
\cite{RoederDiss} for details).


%%%%%%%%%%%%%%%%%%%%%%%%%
\Section{Methods for calculating ordered signatures}

\>NormalSubgroupsForRep( <groupdata>, <divisor> ) O

Let <groupdata> be the output of `PermutationRepForDiffsetCalculations' and 
<divisor> an integer. Then `NormalSubgroupsForRep' calculates all  normal 
subgroups of <groupdata.G> such that the size of the factor group is divisible 
by <divisor> and the factor group is a semidirect product of cyclic groups.

The output is a record consisting of 
\beginlist
 \item{1.} a normal subgroup <.Nsg> of <G>
 \item{2.} the factor group <.fgrp>:=<G>/<Nsg> 
 \item{3.} the epimorphism <.epi> from <G> to <.fgrp>
 \item{4.} a root of unity <.root>
 \item{5.} a galois automorphism <.alpha>
 \item{6.+7.} generators of the factor group <G>/<.Nsg> named <.a> and <.b> 
           such that <.a> is normalized by <.b>.
 \item{8}  a list <.int2pairtable> such that the $i^{th}$ entry ist the pair 
           <[m,n]> with that <Glist[i]^epi=a^(m-1)\*b^(n-1)>
\endlist

<.alpha> and <.root> may be used as input for `OrderedSigs'



\>OrderedSigs( <coeffSums>, <absSum>, <alpha>, <root> ) O

Let $G$ be group which contains a normal subgroup of index $s$ such that 
the coset signature for a difference set for this normal subgroup is 
<coeffSums>. Let $N$ be a normal subgroup of $G$ such that $G/N$ is a 
semidirect product of cyclic group of orders $s,q$  and 
$i$ divides the order of $G/N$. 

Then `OrderedSigs(<coeffSums>,<absSum>,<alpha>,<root>)' calculates 
all ordered signatures for $N$. Here <root> is a primitive $q$-th root 
of unity and <alpha> is a Galois- automorphism of $CS(q)$ with order 
dividing $s$. <absSum> is the order of the difference set.
(i.e. $order=k-\lambda$).

`OrderedSigs' is based on calculations using an $s$-dimensional unitary 
representation of $G/N$. 
In this representation a subset of $G$ induces a semi-circular matrix.
The returned value is a list of lists $s$-tuples
The entries of the $s$-tuples are coefficients of  numbers in 
$\Z[<root>]$ such that the semi-circular matrix defined by these numbers
together with <alpha> meets necessary conditions for matrices induced
by difference sets.
To gain the algebraic numbers from the $s$-tuple <tup>, use 
`List(<tup>,i->CoeffList2CyclotomicList(i,<root>))'

Each $|<coeffSums>|$-tuple returned defines an ordered signature. The ordering
of $G/N$ is chosen to fit to the data returned by `NormalSubgroupsForRep':

$[a^0,a^1,\dots,a^{q-1}],[a^0b,a^1b,\dots,a^{q-1}b],\dots,[a^0b^{s-1},\dots,a^{q-1}b^{s-1}]$



So for the calculation of ordered signatures, smaller ordered
signatures <coeffSums> have to be known. But this is not so bad, as
small signatures are easy to calculate.
The following example shows an application.

\begintt
gap> G:=SmallGroup(273,3);                              
<pc group of size 273 with 3 generators>
gap> Gdata:=PermutationRepForDiffsetCalculations(G);;
gap> CosetSignatures(273,273/3,16);
[ [ 3, 7, 7 ] ]
gap> nsgs:=NormalSubgroupsForRep(Gdata,3);           
[ rec( Nsg := Group([ f2 ]), alpha := ANFAutomorphism( CF(13), 3 ), 
      root := E(13), fgrp := Group([ f1, <identity> of ..., f2 ]), 
      epi := [ f1, f2, f3 ] -> [ f1, <identity> of ..., f2 ], a := f2, 
      b := f1, 
      int2pairtable := [ [ 1, 1 ], [ 1, 2 ], [ 1, 1 ], [ 2, 1 ], [ 1, 3 ], 
...
          [ 8, 3 ], [ 11, 3 ], [ 5, 2 ], [ 11, 3 ] ] ), 
  rec( Nsg := Group([ f3 ]), alpha := ANFAutomorphism( CF(7), 2 ), 
      root := E(7), fgrp := Group([ f1, f2, <identity> of ... ]), 
      epi := [ f1, f2, f3 ] -> [ f1, f2, <identity> of ... ], a := f2, 
      b := f1, 
      int2pairtable := [ [ 1, 1 ], [ 1, 2 ], [ 2, 1 ], [ 1, 1 ], [ 1, 3 ], 
...
          [ 6, 3 ], [ 4, 3 ], [ 4, 2 ], [ 6, 3 ] ] ) ]
gap> osigs:=OrderedSigs([3,7,7],16,nsgs[2].alpha,nsgs[2].root);
[ [ [ 0, 0, 0, 1, 0, 1, 1 ], [ 0, 0, 1, 2, 2, 0, 2 ], [ 2, 2, 0, 2, 0, 0, 1 ] ], 
  [ [ 0, 0, 0, 1, 0, 1, 1 ], [ 0, 1, 2, 2, 0, 2, 0 ], [ 2, 0, 0, 1, 2, 2, 0 ] ], 
...
   [ [ 1, 1, 0, 1, 0, 0, 0 ], [ 2, 2, 1, 0, 0, 2, 0 ], [ 2, 1, 0, 0, 2, 0, 2 ] ] ]
gap> Size(osigs);
98
gap> Set(osigs,g->SortedList(Concatenation(g)));
[ [ 0, 0, 0, 0, 0, 0, 0, 0, 0, 0, 1, 1, 1, 1, 1, 2, 2, 2, 2, 2, 2 ] ]
\endtt

Note that the signature `[3, 7, 7]' can be assumed to be ordered (by
passing to a suitable translate). So even if we are not interested in
*ordered* signatures, we have found out that there is only one admissible
unordered signature for this normal subgroup. To get this result using
`TestedSignatures' would have taken a *very* long time.

Of course, ordered signatures can also be used directly.

\>OrderedSignatureOfSet( set, normal_data ) O

takes a set <set> of integers (meant to be a partial difference set) and
a list of records as returned by `NormalSubgroupsForRep'.
The returned value is a list of lists which is the ordered signature of the
partial difference set <set> and can be compared to the output of `OrderedSigs'



\beginexample
gap> OrderedSignatureOfSet([2,3,4,5],nsgs[2]);  
[ [ 1, 1, 1, 0, 0, 0, 0 ], [ 1, 0, 0, 0, 0, 0, 0 ], [ 0, 0, 0, 0, 0, 0, 0 ] ]
\endexample
