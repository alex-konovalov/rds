% This file was created automatically from about.msk.
% DO NOT EDIT!
\Chapter{About this package}

The {\RDS} package is meant to help with complete searches for
relative difference sets in non-abelian groups. Of course, it also
works for abelian groups, but no special features are implemented for
this case. In particular, there is no support for multipliers.

Furthermore, the focus is on difference sets defining projective
planes. So {\RDS} contains some methods for analyzing projective
planes but has no support for other designs. Nonetheless other designs
may be constructed if they can be described in terms of  difference sets.

{\RDS} has no undocumented functions. While this is generally regarded
as a feature, it leads to a quite long manual and a lot of
documentation not needed for everyday work. So each chapter has a
short introduction helping you to distinguish the ``interesting'' features
from the ``uninteresting'' ones.

For a quick overview, see chapter "RDS:A quick start".

The package is entirely written in {\GAP}. So it should run on every
computer that runs {\GAP}.

%%%%%%%%%%%%%%%%%%%%%%
\Section{Installation}

Just copy the archive to the directory where the other packages live .
For example, {\tt gap/pkg/} in your home directory. Or use the {\tt
  pkg/} directory in one of the paths from the \GAP variable {\tt
  GAP_ROOT}.

Then call \GAP and type

\beginexample
gap> LoadPackage("rds");
----------------------------------------------------------------
Loading  RDS 0.9beta5
by Marc Roeder
For help, type: ?RDS
----------------------------------------------------------------
true
\endexample

For a test, see the example in chapter "RDS:A quick start".


%%%%%%%%%%%%%%%%%%%%%%%%%%%%%%%%%%%%%%%%%%%%%%%%%%
%%
%E

